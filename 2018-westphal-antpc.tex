\documentclass{anstrans}
%%%%%%%%%%%%%%%%%%%%%%%%%%%%%%%%%%%
\title{PyRe: A Cyclus Pyroprocessing Facility Archetype}
\author{Gregory T. Westphal$^*$ and Kathryn D. Huff}

\institute{
Dept. of Nuclear, Plasma and Radiological Engineering, University of Illinois at Urbana-Champaign \\
$^*$gtw2@illinois.edu
}

%%%% packages and definitions (optional)
\usepackage{float}
\floatstyle{plaintop}
\restylefloat{table}
\usepackage{graphicx} % allows inclusion of graphics
\usepackage{booktabs} % nice rules (thick lines) for tables
\usepackage{microtype} % improves typography for PDF
\usepackage{xspace}
\usepackage{tabularx}
\newcommand{\SN}{S$_N$}
\renewcommand{\vec}[1]{\bm{#1}} %vector is bold italic
\newcommand{\vd}{\bm{\cdot}} % slightly bold vector dot
\newcommand{\grad}{\vec{\nabla}} % gradient
\newcommand{\ud}{\mathop{}\!\mathrm{d}} % upright derivative symbol
\newcommand{\Cyclus}{\textsc{Cyclus}\xspace}%
\newcommand{\Cycamore}{\textsc{Cycamore}\xspace}%

\newcolumntype{c}{>{\hsize=.56\hsize}X}
\newcolumntype{b}{>{\hsize=.7\hsize}X}
\newcolumntype{s}{>{\hsize=.74\hsize}X}
\newcolumntype{f}{>{\hsize=.1\hsize}X}
\newcolumntype{a}{>{\hsize=.45\hsize}X}

\usepackage{amsmath}

\usepackage[acronym,toc]{glossaries}
\include{acros}
\makeglossaries
\glsunset{PyRe}
\glsunset{TRU}
\glsunset{LWR}
\glsunset{SNM}
\begin{document}
%%%%%%%%%%%%%%%%%%%%%%%%%%%%%%%%%%%%%%%%%%%%%%%%%%%%%%%%%%%%%%%%%%%%%%%%%%%%%%%%
\begin{abstract}
This work assesses system parameters that influence separation efficiency and 
throughput of pyroprocessing facilities. We leverage these parameters to implement a customizable pyroprocessing facility \emph{archetype}, \gls{PyRe}, for use with the Cyclus framework.
This generic facility model will allow simulations to 
quantify signatures and observables associated with various operational modes 
and material throughputs for a variety of facility designs. Such quantification 
can aid timely detection of material diversion. 
This paper describes the \Glssymbol{PyRe} facility archetype design, pyroprocessing flowsheets captured by the model, and simulation capabilities it enables. 
To analyze data retrieved from the model, we additionally propose a class for tracking and 
observing signatures and observables which will be extensible for other 
facility archetypes in the future.
\end{abstract}
\section{Introduction}
The diversion of significant quantities of \gls{SNM} from the nuclear fuel cycle is major non-proliferation 
concern \cite{noauthor_iaea_2017}. These diversions must be detected in a timely manner using signatures and observables in 
order to properly safegaurd the fuel cycle. Timely detection is critical in non-proliferation to discover these shadow fuel cycles
before diverted material is further processed. Pyroprocessing is a used nuclear fuel separations technology for advanced reactors. 
Signatures and observables are used to detect diversion of nuclear material.
The goal of this research is to identify potential signs of material diversion in a pyroprocessing facility and implement models 
of these processes into a detailed pyroprocessing facility archetype to the modular, agent-based, fuel cycle simulator, \Cyclus \cite{huff_fundamental_2016}. This facility archetype will equip users of the \Cyclus fuel cycle simulator to investigate 
detection timeliness enabled by measuring signatures and observables in various fuel cycle scenarios.
%%%%%%%%%%%%%%%%%%%%%%%%%%%%%%%%%%%%%%%%%%%%%%%%%%%%%%%%%%%%%%%%%%%%%%%%%%%%%%%%
\section{Background: \Cyclus}
\Cyclus models the flow of material through user-defined nuclear fuel cycle scenarios. Facilities in nuclear fuel cycles vary, 
requiring a diverse collection of pre-designed facility process models, known as \emph{archetypes}. \Cycamore, the CYClus 
Additional MOdules REpository, provides common facility archetypes (separations, enrichment, reactor, etc.)
\cite{carlsen_cycamore_2014}. Archetypes are customizable agent models which populate the simulation. Exact isotopes are dynamically tracked between facilities in discrete time steps \cite{huff_fundamental_2016}. This work seeks to add signature and observable tacking capabilities to \Cyclus. Potentially trackable signatures and observables include truck deliveries and  power draw  \cite{Hou_2016,Yilmaz_2016}. This list is expanded upon in Table \ref{tab:params} to include pyroprocessing parameters.

%%%%%%%%%%%%%%%%%%%%%%%%%%%%%%%%%%%%%%%%%%%%%%%%%%%%%%%%%%%%%%%%%%%%%%%%%%%%%%%%
\section{Pyroprocessing}
Pyroprocessing is an electrochemical separation process used to recycle spent fuel into metallic fuel for use in advanced reactors.
Separation efficiencies differ according to pyroprocessing facility design and fuel type. There are four major 
pyroprocessing systems with observable waste: voloxidation, electroreduction, electrorefining, and electrowinning \cite{Borrelli_2017}.  

\begin{figure}[ht] % replace 't' with 'b' to force it to be on the bottom
	\centering
	\includegraphics[width=0.5\textwidth]{flowchart}
	\caption{An archetype design flowchart of pyroprocessing facilities including observable outputs and \Cyclus variables.}
	\label{fig:flowchart}
\end{figure}

Figure \ref{fig:flowchart} demonstrates the primary separations steps involved in a general pyroprocessing facility. Main process parameters are placed to the left of their respective subprocess. Boxes on the 
right side of the processes contain the observable waste produced by each step that \gls{PyRe} tracks. 
The explicit behavior of each main process is described below. 

\subsection{Voloxidation}

\gls{LWR} fuel must be treated and separated before proceeding with electrolytic processes. Uranium dioxide heated to 
500$^{\circ}$C is converted to $U_3O_8$ while noble gases, carbon, and tritium are collected to decay in storage. 
Actinides are also converted to their stable oxide forms and a majority are removed \cite{flowsheet_1998,jubin_spent_2009}. 
Heating uranium dioxide above 800$^{\circ}$C increases voloxidation throughput.
Cycling oxidants between H$_2$ and air also improves the U$_3$O$_8$ reaction rate \cite{jubin_spent_2009}.

\begin{figure}[ht]
	\centering
	\includegraphics[width=\linewidth]{volox}
	\caption{A material balance over the voloxidation sub-process, including observables.}
	\label{fig:volox}
\end{figure}

\subsection{Electroreduction}

The oxidant is converted into metallic fuel through electroreduction to be further refined through electrorefining and electrowinning. 
Yellowcake, created in voloxidation, enters the cathode, a negatively charged metal basket. 
A current density between 100 and 500 mA/cm$^2$ is applied to the anode in a molten LiCl salt. 
The electrolytic reduction process primarily results in diffusion of Cs, Ba and Sr, along with reduction and conversion of zirconium into metallic form \cite{choi_electrochemical_2015,flowsheet_1998}.
Electroreduction can further improve its throughput by adding Li$_2$O as a catalyst; this catalyst also prevents dissolution 
of the anode \cite{choi_electrochemical_2015}. Since Li$_2$O is used to speed up the reaction,
the operators could add more oxide than reported to \gls{IAEA}. More frequent shipments 
of lithium oxide can be tracked as an observable to match records.

\begin{figure}[ht]
	\centering
	\includegraphics[width=\linewidth]{reduction}
	\caption{A material balance over the electroreduction process with signatures and observables.}
	\label{fig:reduction}
\end{figure}

This stream has a considerable decay signature proportional to the efficiency and size of the feed batch \cite{Borrelli_2017,flowsheet_1998}.

\subsection{Electrorefining}

Once in metallic form, electrorefining electrochemically separates uranium and  for fuel fabrication.
The uranium and salt mixture from reduction is fed into an anode basket suspended in a graphite cathode. 
A LiCl-KCl eutectic is used as an electrolyte above 500$^{\circ}$C \cite{flowsheet_1998,lee_korean_2011}. 
Uranium dissolves at the anode to recombine at the cathode as metallic uranium.
\glsreset{TRU} 
Waste \glspl{TRU} and lanthanides are in a soluble chloride form  while fission products and cladding remain in the anode
basket. Finally, actinides and fission products are removed from the cladding electrochemically \cite{lee_korean_2011}.

Lee et al. \cite{lee_advanced_2008} show decreasing system pressure improves removal efficiency. 
Temperature, however, exhibits the opposite effect: as temperature decreases so does salt removal. This comes into effect 
particularly depending on material choice of instrumentation and containment \cite{lee_advanced_2008}. 
Iron, for example, limits operating temperature because a eutectic forms at 725$^{\circ}$C \cite{chapman_revision_1984}.
In facilities where iron equipment is present, temperatures are limited to 700$^{\circ}$C, hindering efficiency. 
Cathode arrangement and anode rotation speed also affect the collection of uranium 
dendrites \cite{lee_advanced_2008}. A central stirrer mixes uranium dendrites stuck on 
the vessel, improving separation efficiency and increasing throughput. 

\begin{figure}[ht]
	\centering
	\includegraphics[width=1\linewidth]{refining}
	\caption{A material balance over the electrorefining sub-process, including observables.}
	\label{fig:refining}
\end{figure}
 
The electrorefining process also produces a fission product waste stream which requires monitoring. 
The following products are produced and tracked in \gls{PyRe} at this step: Tc, Ag, Pd, Rh, Ru, Mo and Zr \cite{flowsheet_1998}. 
Uranium and \gls{TRU} product streams separated at this stage are sent to fuel fabrication, while the remaining salt is reformed as an oxidant and recirculated.
Separation efficiencies are taken after recirculation, and treated as a once-through cycle. Cyclus' time step
is not detailed enough to benefit from analyzing intra-batch processes. Therefore, only end-state efficiencies are used rather than an explicit model.

\subsection{Electrowinning}

Molten salt containing \glspl{TRU} from electrorefining is separated through electrowinning. This process separates trace uranium quantities, lanthanides and fission products. 
At 500$^{\circ}$C there is approximately 99 wt\% reduction in actinides and lanthanides \cite{flowsheet_1998}. 
Throughput also depends on material choice for the inert electrodes, impacting separation 
efficiency \cite{koyama_development_2012}. A shroud surrounds the anode to provide a path for O$^{2-}$ ions to the anode and 
prevent Cl$_2$ from corroding the anode \cite{kim_development_2013,choi_electrochemical_2015}. Optimum operating current 
depends on material choice for the anode shroud since a nonporous shroud limits ion pathways to the anode contact points.
Higher porosity corresponds to free ion paths and a higher current. Increased currents reduce the separation time for electroreduction and electrowinning \cite{choi_electrochemical_2015}.

\begin{figure}[ht]
	\centering
	\includegraphics[width=1\linewidth]{winning}
	\caption{A material balance over the electrowinning sub-process, including observables.}
	\label{fig:winning}
\end{figure}

Figure \ref{fig:winning} shows that electrowinning product recirculates to electrorefining after removing lanthanides.
Fission products remaining in the salt from electrorefining are also removed here. In addition to physically tracked 
quantities, facility power can also be monitored to observe the use of current to increase throughput. 
%%%%%%%%%%%%%%%%%%%%%%%%%%%%%%%%%%%%%%%%%%%%%%%%%%%%%%%%%%%%%%%%%%%%%%%%%%%%%%%%
\section{Method: \Cyclus Simulation}
The separations facility provided by the \Cycamore library is used as an 
initial model of a simple \gls{PRIDE} facility. 
Users provide the separations archetype with a feed stream and facility efficiencies.  
Each waste stream requires a material balance over voloxidation, electroreduction, electrorefining and electrowinning. Main 
waste streams are metallic waste, ceramic waste from electrowinning and electroreduction, and vitrified waste. Vitrified 
waste contains the majority of \Glssymbol{TRU}s, Sr, and rare-earth elements. Elemental separation efficiencies are 
determined through theoretical material balance determined by the NEA and Hermann et al \cite{flowsheet_1998,herrmann_separation_2010}. 
The simple simulation using a separations facility was run to verify the table of efficiencies input to \Cyclus. A scenario was 
created of one separations facility with a feed of five year cooled spent \Gls{LWR} 
fuel at a burn-up of 45 \gls{GWd} per \gls{MTIHM} to 
match results seen in \cite{flowsheet_1998}.

A pyroprocessing facility can be modeled with the separations archetype at low fidelity. 
The goal for \Gls{PyRe} is to include facility configuration parameters and provide the user with data to optimize detector placement.
\Gls{PyRe} accomplishes this by performing detailed separations at each subprocess and compiling the resulting streams.
Any material not separated by the four processes is categorized as leftover commodity, ensuring material conservation is maintained. 
Unit tests and runtime checks will confirm material conservation. 
Efficiencies also vary according to the feed stream resulting in different waste streams for \Gls{LWR} and FR fuels, for example. 

\subsection{Parameters}
Facility configuration parameters customize the pyroprocessing archetype and vary by design. 
These pyroprocessing designs vary in multiple aspects which affect the throughputs and efficiencies of different waste streams. 
Variables have been described above through material balance of each sub-process and their system parameters. 
Table \ref{tab:params} consists of the  derived from the previously mentioned process parameters and material flow. 

\begin{table}[h]
	\centering
	\begin{tabularx}{0.5\textwidth}{llcr}
		\hline
		\textbf{Sub-process} & \textbf{Parameters} & \textbf{S \& O} & \textbf{Refs} \\
		\hline
		Voloxidation & Volume & Tritium & \cite{jubin_spent_2009} \\
		& Oxidant & $^{14}$C & \cite{flowsheet_1998} \\
		& Flow Rate &  $^{129}$I &  \\
		& Temperature & $^{85}$Kr &  \\
		& Time & Actinides & \\ \hline
		Electroreduction & Volume & $^{90}$Sr & \cite{Borrelli_2017} \\
		& Batch Size & $^{135}$Cs & \cite{flowsheet_1998} \\
		& Li$_2$O wt\% & $^{137}$Cs & \cite{choi_electrochemical_2015} \\
		& Current & Power Draw & \cite{lee_korean_2011} \\
		& Porosity & Shipments & \cite{lee_modeling_2016} \\
		& Distillation Speed & Throughput & \\ 
		& Time & & \\ \hline
		Electrorefining & Volume & Fission Products & \cite{lee_advanced_2008} \\
		& Time & Power Draw & \cite{lee_korean_2011} \\
		& Material & Waste Salt & \cite{flowsheet_1998} \\
		& Anode Rotation & Vacuum Pressure & \cite{koyama_development_2012} \\
		& Stirrer Speed & Temperature & \cite{kim_development_2013} \\
		& Pressure & Throughput & \\
		& Temperature & & \\ \hline
		Electrowinning & Current & Power Draw & \cite{flowsheet_1998} \\
		& Shroud Material & Cadmium Waste & \cite{lee_korean_2011} \\
		& Time & Fission Products & \cite{Borrelli_2017} \\
		& Flow Rate & Lanthanides & \\
		&  & $^{135}$Cs & \\
		&  & $^{137}$Cs & \\ \hline
		Facility & Throughput & Shipments & \\
		& Batch Size & Parking Lot & \\
		& & Thermal Image & \\
		\hline
	\end{tabularx}
	\caption {Archetype inputs and signatures \& observables at each sub-process.}
	\label {tab:params}
\end{table}

Signatures and observables contain quantities measured directly inside
the facility and indirect characteristics observables at distance. A broader category for the facility as a whole is also described for global parameters such as
throughput and batch size. Since throughput is a facility observable, 
it is seen in a majority of the sub-processes. Reduction is limited by 
batch size therefore reducing the throughput of proceeding steps to the 
electrochemical process. The finished product and cars in the parking lot also serve as an indicator to excess work being done. Thermal imaging, further, can determine the operational status of the facility. 

%%%%%%%%%%%%%%%%%%%%%%%%%%%%%%%%%%%%%%%%%%%%%%%%%%%%%%%%%%%%%%%%%%%%%%%%%%%%%%%%

\section{Discussion}
Using a material balance area over electrorefining and electroreduction yields the majority of detectable waste from 
the electrochemical processes.  Material balances over the remaining 
processes are used to verify diversion did not occur. Fuel fabrication is also at high risk of diversion. Finished product can be diverted with no additional 
processing steps and a material balance area is also taken here. 

Multiple scenarios must be considered to determine the most sensitive points for diversion. 
Each facility parameter must be varied to observe their effects, as well as using a limited number of material balance areas. 
Scenarios will be run that include various monitoring points with the goal of determining if excess material was produced 
and divertyed. 

For example, an increase in Cs production points to electroreduction and electrowinning. 
If both increase similarly then current is likely affected as these processes share an increase in efficiency with increased 
current. Further in this scenario, if Cs production increases while Sr does not, electrowinning must be the point at which 
parameters are altered. A set of these scenarios will be used for sensitivity and importance analysis on the generic 
pyroprocessing facility.

%%%%%%%%%%%%%%%%%%%%%%%%%%%%%%%%%%%%%%%%%%%%%%%%%%%%%%%%%%%%%%%%%%%%%%%%%%%%%%%%
\section{Conclusions}
This analysis demonstrates the variability in pyroprocessing facilities and their affects on potential 
signatures and observables that will be tracked through a detailed archetype. \Cyclus also is outlined as a tool in 
detecting shadow fuel cycles through its agent-based simulation and modular facilities, allowing for variations in 
plant design. Modeling and simulation of shadow fuel cycles will be performed in the \Cyclus environment after 
creation of a library specific to the unique needs of electrochemical refinement. Data from these simulations with 
additional signatures and observables will inform detector placements and measurement points leading to more 
reliable diversion detection.

%%%%%%%%%%%%%%%%%%%%%%%%%%%%%%%%%%%%%%%%%%%%%%%%%%%%%%%%%%%%%%%%%%%%%%%%%%%%%%%%
\section{Acknowledgments}
This material is based upon work supported by the Department of Energy
National Nuclear Security Administration under Award Number(s) DE-NA0002576 via
the Consortium for Nonproliferation Enabling Capabilities.

%%%%%%%%%%%%%%%%%%%%%%%%%%%%%%%%%%%%%%%%%%%%%%%%%%%%%%%%%%%%%%%%%%%%%%%%%%%%%%%%
\bibliographystyle{ans}
\bibliography{bibliography}
\end{document}

