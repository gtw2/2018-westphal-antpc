\documentclass{anstrans}
%%%%%%%%%%%%%%%%%%%%%%%%%%%%%%%%%%%
\title{Signatures and Observables of the Nuclear Fuel Cycle}
\author{\textbf{Gregory T. Westphal}, and Kathryn D. Huff}

\institute{
Dept. of Nuclear, Plasma and Radiological Engineering, University of Illinois at Urbana-Champaign \\
gtw2@illinois.edu
}

%%%% packages and definitions (optional)
\usepackage{graphicx} % allows inclusion of graphics
\usepackage{booktabs} % nice rules (thick lines) for tables
\usepackage{microtype} % improves typography for PDF
\usepackage{xspace}
\usepackage{tabularx}
\newcommand{\SN}{S$_N$}
\renewcommand{\vec}[1]{\bm{#1}} %vector is bold italic
\newcommand{\vd}{\bm{\cdot}} % slightly bold vector dot
\newcommand{\grad}{\vec{\nabla}} % gradient
\newcommand{\ud}{\mathop{}\!\mathrm{d}} % upright derivative symbol
\newcommand{\Cyclus}{\textsc{Cyclus}\xspace}%
\newcommand{\Cycamore}{\textsc{Cycamore}\xspace}%
\newcolumntype{c}{>{\hsize=.56\hsize}X}
\newcolumntype{b}{>{\hsize=.7\hsize}X}
\newcolumntype{s}{>{\hsize=.74\hsize}X}
\newcolumntype{f}{>{\hsize=.1\hsize}X}
\newcolumntype{a}{>{\hsize=.45\hsize}X}

\begin{document}
%%%%%%%%%%%%%%%%%%%%%%%%%%%%%%%%%%%%%%%%%%%%%%%%%%%%%%%%%%%%%%%%%%%%%%%%%%%%%%%%
\section{Introduction}
The diversion of significant quantities of special nuclear material from the nuclear fuel cycle is major non-proliferation concern. These diversions must be detected in a timely manner using signatures and observables in order to properly safegaurd the fuel cycle. Pyroprocessing is an up and coming reprocessing technology capable of both converting current generation waste into molten salt fuel, and reprocessing next generation molten salt fuel types. With a new reprocessing technology comes new signatures and observables which in turn necessitate new diversion detection methods. The goal of this research is to identify potential signs of pyroprocessing diversion and implement models of these processes into a detailed pyroprocessing facility archetype to the modular, agent-based, fuel cycle simulator, \Cyclus. This facility archetype will equip users of the \Cyclus fuel cycle simulator to investigate the detection timeliness enabled by novel signatures and observables in various fuel cycle diversion scenarios.

%%%%%%%%%%%%%%%%%%%%%%%%%%%%%%%%%%%%%%%%%%%%%%%%%%%%%%%%%%%%%%%%%%%%%%%%%%%%%%%%
\section{Background: \Cyclus}
\Cyclus models the flow of material through agent-based user-defined nuclear fuel cycles. Facilities in nuclear fuel cycles vary, requiring a diverse collection of pre-designed facilities, archetypes. \Cycamore, \Cyclus Additional Modules Repository, provides the common facilities seen simulations (separations, enrichment, reactor, etc.). Archetypes provide a mold for users to input values specific to each facility with a known output \Cyclus can interpret. Simulations run in discrete time steps that allow exact isotopes and respective quantities to be tracked in time between regions or facilities \cite{huff_fundamental_2016}. Tracking requires a form of signature or observable to follow the material accurately throughout an institution or cycle. Current capabilities would be truck deliveries, power draw, and 'smoke' production that have been proven sufficient for maximum likelihood estimations of diversion \cite{Hou_2016,Yilmaz_2016}.
\Cyclus' discrete time allows users to investigate diversion and flag the time and location diversion occurs.

%%%%%%%%%%%%%%%%%%%%%%%%%%%%%%%%%%%%%%%%%%%%%%%%%%%%%%%%%%%%%%%%%%%%%%%%%%%%%%%%
\section{Background: Pyroprocessing}
Pyroprocessing is an eletrochemical separation process used to recycle spent fuel into molten salt fuel. With the capability of processing various forms of waste, efficiencies will differ according to design. There are four major systems within pyroprocessing with observable waste: voloxidation, electroreduction, electrorefining, electrowinning \cite{Borrelli_2017} . These processes have been defined by KAERI through their development of PRIDE (Pyroprocessing Inactive integrated Demonstration) facility. 

\subsection{Voloxidation}

For LWR fuel, the fuel must be initially treated and separated before proceeding with electrolytic processes. Heated under 500$^{\circ}$C, noble gases and tritium are collected to decay in storage, and uranium dioxide is converted to $U_3O_8$. Actinites are also converted to their stable oxide forms and a majority are removed. The result cladding must undergo electrowinning for removal of remaining oxides \cite{flowsheet_1998}. 

\subsection{Electroreduction}

The waste stream enters the cathode metal basket as pellets of oxides created in voloxidation. Between 100 and 500 mA/cm$^2$ is run through the anode, typically platinum, in a molten LiCl salt electrolyte. Li$_2$O is used as a catalyst and prevents dissolution of the platinum anode \cite{choi_electrochemical_2015}. The catalyst often is used in concentration of 1 wt\% with potential for up to 3 wt\%. Since Li$_2$O is used to speed up the reaction, it is important to note that for signatures and observables the operators could add more oxide than reported to IAEA. More frequent shipments of lithium oxide can be tracked as an observable to match records. The electrolytic reduction process results in the diffusion of Cs, Ba and Sr primarily, along with the reduction and conversion of zirconium into metallic form \cite{choi_electrochemical_2015,flowsheet_1998}.

\subsection{Electrorefining}

Recoverable waste from reduction is fed into an anode basket suspended in a graphite cathode. LiCl-KCl eutectic is used as electrolyte above 500$^{\circ}$C. UCl$_3$ is added at 6 wt\% and potential is again run across the anode \cite{flowsheet_1998,lee_korean_2011}. The uranium dissolves at the anode to recombine at the cathode as metallic uranium. The waste transuranics (TRUs) and lathanides are in a soluble chloride form  while fission products and cladding remain in the anode basket. Finally, actinides and fission products are removed from the cladding electrochemically \cite{lee_korean_2011}.

\subsection{Electrowinning}

The molten salt contains TRUs from electrorefining and are separated through electrowinning with trace uranium quantities. The mixture is placed on a solid cathode along with chlorine gas at a graphite anode. With a temperature of 500$^{\circ}$C there is approximately 99 wt\% reduction in actinides and lanthanides. 
%%%%%%%%%%%%%%%%%%%%%%%%%%%%%%%%%%%%%%%%%%%%%%%%%%%%%%%%%%%%%%%%%%%%%%%%%%%%%%%%
\section{Method: \Cyclus Simulation}
The separations facility provided by the \Cycamore library expansion is used as an initial model of a simple PRIDE facility. The separations archetype allows for the declaration of a feed stream and requires the user-definition of facility efficiencies. Each waste stream requires a material balance determined from the processes mentioned previously. The main waste streams found are metallic waste, ceramic waste from electrowinning and electroreduction, and vitrified waste. Vitrified waste contains the majority of TRUs, Sr, and rare-earth elements. The efficiencies of each stream and their isotopes are determined through theoretical material balance determined by \cite{flowsheet_1998}. The simple simulation was run to verify the table of efficiencies input to \Cyclus.

Pyroprocessing, however, is not identical to a separations facility and requires a separate library, or archetype. The goal for this archetype is to include the variations possible in facility configuration and their respective effects on the efficiency table. Efficiencies also vary according to the feed stream resulting in different waste streams for LWR and FR fuels, for example. Multiple material choices exist for anodes and cathodes as well as other design choices that need to be taken into account. 

\subsection{Variations}

Advanced methods for electrorefining developed by KAERI \cite{lee_advanced_nodate} further improve salt removal efficiencies and explore the effect of parameter variation. Temperature and pressure provide a significant improvement in removal efficiency, but can vary depending on facility specifications. 

\begin{table}[h]
	\centering
	\begin{tabularx}{0.5\textwidth}{ccc}
		\hline
		\textbf{Vapor Pressure (mTorr)} & \textbf{Evaporation Coefficient} & \textbf{Salt Removal Efficiency} \\
		\hline
		500 & 3.04x10$^{-5}$ & 97.3 \\
		300 & 2.66x10$^{-5}$ & 99.6 \\
		200 & 1.25x10$^{-4}$ & 99.4 \\
		100 & 1.78x10$^{-4}$ & 99.9 \\
		\hline
	\end{tabularx}
	\caption {Under vacuum pressure, the evaporation coefficient and theoretical max removal efficiencies are calculated \cite{lee_advanced_nodate}}
	\label {tab:pressure}
\end{table}

As shown by Table I, the addition of vacuum pressure to the system improves removal efficiency with a noticeable increase between 500 and 300 mTorr. Temperature, however, exhibits the opposite effect: as temperature decreases so does salt removal substantially. This comes into effect particularly depending on material choice. The most limiting being iron, as a eutectic forms between Fe and U at 725$^{\circ}$C \cite{chapman_revision_1984}.

\begin{table}[h]
	\centering
	\begin{tabularx}{0.5\textwidth}{ccc}
		\hline
		\textbf{Temperature (C)} & \textbf{Evaporation Coefficient} & \textbf{Salt Removal Efficiency} \\
		\hline
		1000 & 1.25x10$^{-4}$ & 99.4 \\
		900 & 5.62x10$^{-5}$ & 98.8 \\
		800 & 4.63x10$^{-5}$ & 94.9 \\
		700 & 3.13x10$^{-6}$ & 82.4 \\
		\hline
	\end{tabularx}
	\caption {The evaporation coefficient and removal efficiencies are again calculated for varying temperature \cite{lee_advanced_nodate}}
	\label {tab:temperature}
\end{table}

If the facility is limited below 700$^{\circ}$C in the case of iron equipment, the efficiency is significantly hindered. In these advanced processes multiple cathodes are placed surrounding an agitating central anode. Cathode arrangement and anode rotation speed also affect the collection of uranium dendrites. Uneven, or sub-optimal, placement of cathodes result in an uneven electric field for electrolysis and lower efficiency while a high rotation speed causes remixing \cite{lee_advanced_nodate}.

Inert electrodes require various voltages depending on material choice 
%%%%%%%%%%%%%%%%%%%%%%%%%%%%%%%%%%%%%%%%%%%%%%%%%%%%%%%%%%%%%%%%%%%%%%%%%%%%%%%%
\section{Method: Proposed Algorithms}
In Best Practices of Scientific computing \cite{wilson_best_2014}, Wilson et 
al. highlights the fact that similar to building experimental apparatus, 
constructing software requires careful building and validation to ensure their 
reliability. Furthermore, Wilson et al discusses that because software is 
commonly reused, it can result in a negative long term effect on the integrity 
of the group's work if a bug is not found. 

An important practice for verification and maintenance of code is to write and 
run tests. Automated tests ensure that a piece of code 
is functioning the way it is intended. The most basic test is the unit test 
which refers to the testing of a single function \cite{wilson_best_2014}. For 
the demand-driven deployment algorithms, unit tests will be written for each 
section of code to ensure all components of the code are reliable. Testing may 
not be perfect at capturing all the bugs; however, it minimizes them. 

The goal is for the demand-driven deployment algorithms to be integrated with 
the \Cyclus framework in the long term and used to optimize simulations of 
transition analyses. Therefore, if the algorithms are not well tested, they may 
have undiagnosed logic flaws or bugs. This would result in inaccuracies with 
the conclusions drawn from the transition analyses simulations and further 
compromise any experiments that use the algorithm in the future 
\cite{wilson_best_2014}.    

\begin{figure}[ht] % replace 't' with 'b' to force it to be on the bottom
	\centering
	\includegraphics[width=0.35\textwidth]{Hou_Network}
	\caption{An array of facilities with shadow interactions proposed by Hou et al\cite{Hou_2016}}
	\label{fig:maximumlikelihood}
\end{figure}

%%%%%%%%%%%%%%%%%%%%%%%%%%%%%%%%%%%%%%%%%%%%%%%%%%%%%%%%%%%%%%%%%%%%%%%%%%%%%%%%
\section{Conclusions}
The above sections outline the types of numerical experiments that will be 
implemented to test the non-optimizing prediction algorithm for the 
Demand-Driven \Cycamore Archetype project. Implementation of these and further 
tests will ensure the reliability of the prediction algorithms.  
At present, iterative feedback between the testing team at the University of 
Illinois and the archetype development team at the University of South Carolina 
is driving targeted development of the non-optimizing agent archetype so that 
it will successfully pass the tests. 

%%%%%%%%%%%%%%%%%%%%%%%%%%%%%%%%%%%%%%%%%%%%%%%%%%%%%%%%%%%%%%%%%%%%%%%%%%%%%%%%
\section{Acknowledgments}
This research was performed using funding received from the Consortium for Nonproliferation Enabling Capabilities under award number 1-483313-973000-191100.

%%%%%%%%%%%%%%%%%%%%%%%%%%%%%%%%%%%%%%%%%%%%%%%%%%%%%%%%%%%%%%%%%%%%%%%%%%%%%%%%
\bibliographystyle{ans}
\bibliography{bibliography}
\end{document}

